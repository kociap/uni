\documentclass[a4paper, 12pt]{article}

\usepackage[T1]{fontenc}
\usepackage{amssymb}
\usepackage{amsthm}
\usepackage{float}
\usepackage{booktabs}
\usepackage{mathtools}

\let\emptyset\varnothing
\renewcommand\qedsymbol{$\blacksquare$}

\DeclarePairedDelimiter{\abs}{\lvert}{\rvert}

\begin{document}

\title{
  Języki Formalne i Techniki Translacji \\
  \vspace{0.5em}
  \begin{center}\Large Lista 4, Zadanie 7\end{center}}
\author{Piotr Kocia}
\date{}
\maketitle

We are to prove that the language $L$ over the alphabet $\{ 1, 2, 3, 4 \}$ such
that $L = \{ w \ | \ \abs{w}_1 = \abs{w}_2 \land \abs{w}_3 = \abs{w}_4 \}$ is
either context-free or not. We begin by assuming it is not context-free and use
proof by contradiction as this allows us to use the pumping lemma for
context-free languages.

\begin{proof}
Assume $L$ is context-free. By the pumping lemma let $p$ be the pumping length,
then let $s = 1^p3^p2^p4^p$. Then there exists no partition of $s$ into $uvwxy$
such that $\abs{vwx} \leq p$ and $\abs{vw}_1 = \abs{vw}_2 \land \abs{vw}_3 =
\abs{vw}_4$.
\end{proof}

The condition that $vw$ contains the same number of 1,2 and 3,4 symbols is
necessary for when we apply the pumping part of the lemma, that is $uv^nwx^ny, n
\geq 0$, the lemma produces words that do not satisfy the conditions of the
language - if $\abs{vw}_1 \neq \abs{vw}_2$, then while pumping we produce
different quantites of 1s and 2s, hence failing the condition.

\end{document}
