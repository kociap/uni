\documentclass[12pt, a4paper]{article}

\usepackage[T1]{fontenc}
\usepackage{courier}

\usepackage[left=3cm, right=3cm, top=4cm, bottom=4cm]{geometry}
\usepackage{amssymb}
\usepackage{amsmath}
\usepackage{booktabs}
\usepackage{setspace}
\usepackage{float}
\usepackage{mathtools}
\usepackage{makecell}
\usepackage{graphicx}
\usepackage{listings}
\usepackage[usenames,dvipsnames]{xcolor}
\usepackage{hyperref}
\usepackage{graphicx}
\graphicspath{./}

\newcommand{\code}[1]{\texttt{#1}}

\newcommand{\R}{\mathbb{R}}

\DeclarePairedDelimiter\abs{\lvert}{\rvert}

\onehalfspacing

\title{
  Obliczenia Naukowe\\
  \begin{center}\Large Lista 4\end{center}
}
\author{Piotr Kocia}

\begin{document}

\maketitle

\tableofcontents

\section{Introduction}
In the following text we will encounter two concepts - interpolation using
Newton's polynomial and Horner's polynomial evaluation. We intend to explain
those prior to .

\subsection{Interpolation and Newton's polynomial}
We are faced with the problem of finding new data points based on our initial
discrete set of data. We would like to find a function that approximates the
dataset in a certain interval. This problem is known as interpolation. We can
neatly approximate, or interpolate, a function in an interval using a
polynomial. It is obvious that we may approximate a function using a polynomial
of degree 0
$$
p(x) = a_0 = f(x_0)
$$
however, this approximation is far from ideal as it approximates only one point
(unless the function is a constant function). We would like to add another point
to our interpolation polynomial so that we can approximate a more than just a
horizontal line. We thus add a term $a_1(x - x_0)$ which is 0 at $x_0$, leaving
us with the original point $a_0$, and non-zero everywhere else. $p(x)$ then
becomes
$$
p(x) = a_0 + a_1(x - x_0)
$$
We do not yet know what $a_1$ is, but we can easily derive it. If we evaluate $p$
at $x_1$, it should yield the same result as $f(x_1)$, hence $f(x_1) = f(x_0) +
a_1(x_1 - x_0)$ and thus
$$
a_1 = \frac{f(x_1) - f(x_0)}{x_1 - x_0}
$$
We may continue to add points to the polynomial to improve the approximation. We
would add a third point by adding a term $a_2(x - x_0)(x - x_1)$ to make
$$
p(x) = a_0 + a_1(x - x_0) + a_2(x - x_0)(x - x_1)
$$
We note that we have now two terms involving x which evaluate to 0 at $x_0$ and
$x_1$, respectively. That is because $p$ already gives us the correct results at
those points and we would like to maintain that property. $a_2$ may be derived
similarly to $a_1$
\begin{gather}
  y_2 = a_0 + a_1(x_2 - x_0) + a_2(x_2 - x_0)(x_2 - x_1) \nonumber \\
  y_2 = y_0 + \frac{y_1 - y_0}{x_1 - x_0}(x_2 - x_0) + a_2(x_2 - x_0)(x_2 - x_1) \nonumber \\
  y_2 - y_0 - \frac{y_1 - y_0}{x_1 - x_0}(x_2 - x_0) = a_2(x_2 - x_0)(x_2 - x_1) \nonumber \\
  y_2 - y_1 + y_1 - y_0 - \frac{y_1 - y_0}{x_1 - x_0}(x_2 - x_0) = a_2(x_2 - x_0)(x_2 - x_1) \nonumber \\
  \frac{y_2 - y_1}{x_2 - x_1}(x_2 - x_1) + \frac{y_1 - y_0}{x_1 - x_0}(x_1 - x_0) - \frac{y_1 - y_0}{x_1 - x_0}(x_2 - x_0) = a_2(x_2 - x_0)(x_2 - x_1) \nonumber \\
  \frac{y_2 - y_1}{x_2 - x_1}(x_2 - x_1) - \frac{y_1 - y_0}{x_1 - x_0}(x_2 - x_1) = a_2(x_2 - x_0)(x_2 - x_1) \nonumber \\
  \frac{y_2 - y_1}{x_2 - x_1} - \frac{y_1 - y_0}{x_1 - x_0} = a_2(x_2 - x_0) \nonumber \\
  \frac{\frac{y_2 - y_1}{x_2 - x_1} - \frac{y_1 - y_0}{x_1 - x_0}}{x_2 - x_0} = a_2 \label{eq:newton}
\end{gather}
Let us introduce notation
\begin{alignat*}{2}
  f[x_k] &= y_k \quad && k \in \{0, ..., n\} \\
  f[x_k, ..., x_{k+j}] &= \frac{f[x_{k+1}, ..., x_{k+j}] - f[x_k, ..., x_{k+j+1}]}{x_{k+j} - x_k}, \ && k \in \{0, ..., n - j\}, j \in \{1, ..., n\}
\end{alignat*}
Then equation \ref{eq:newton} becomes
\begin{gather*}
  \frac{f[x_1, x_2] - f[x_0, x_1]}{x_2 - x_0} = a_2 \\
  f[x_2, x_1, x_0] = a_2
\end{gather*}
This polynomial, called the Newton's polynomial, may be generalised to any
degree by adding more terms.

\subsection{Horner's algorithm}
Given a polynomial
$$
p(x) = a_0 + a_1x + a_2x^2 + a_3x^3 + ... + a_nx^n
$$
we may rewrite it as
$$
p(x) = a_0 + x(a_1 + x(a_2 + x(a_3 + ... + x(a_{n-1} + xa_n))))
$$
This formula, when programmed into a computer, requires only n multiplications
and n additions to be fully evaluated, which is optimal. The implementation may
be easily provided as the algorithm consists of repeated multiplication and
addition.

\section{Problem 5}
We are to plot the following functions and their Newton interpolation
\begin{itemize}
  \item $e^x$ in $[0, 1]$, $n = 5, 10, 15$,
  \item $x^2 \sin x$ in $[-1, 1]$, $n = 5, 10, 15$.
\end{itemize}

\subsection{Results}
The results for the first function are presented in Figure \ref{fig:ex5_f1_5},
Figure \ref{fig:ex5_f1_10} and Figure \ref{fig:ex5_f1_15}, and for the second
function in Figure \ref{fig:ex5_f2_5}, Figure \ref{fig:ex5_f2_10} and Figure
\ref{fig:ex5_f2_15}.

\subsection{Conclusions}
The interpolation of both functions is accurate as it closely matches the
curves. Interpolation works well with this particular kind of smooth curves.

\section{Problem 6}
We are to plot the following functions and their Newton interpolation
\begin{itemize}
  \item $\abs{x}$ in $[-1, 1]$, $n = 5, 10, 15$,
  \item $\frac{1}{1+x^2}$ in $[-5, 5]$, $n = 5, 10, 15$.
\end{itemize}

\subsection{Results}
The results for the first function are presented in Figure \ref{fig:ex6_f1_5},
Figure \ref{fig:ex6_f1_10} and Figure \ref{fig:ex6_f1_15}, and for the second
function in Figure \ref{fig:ex6_f2_5}, Figure \ref{fig:ex6_f2_10} and Figure
\ref{fig:ex6_f2_15}.

\subsection{Conclusions}
Accurate interpolation of the first function is in general impossible. The
function is not differentiable at $x = 0$ (has a sharp peak) and thus is
difficult to approximate with a smooth (differentiable) curve at that point. On
the other hand, the second function does not suffer from this problem and the
accuracy of the approximation improves with the degree of the interpolating
polynomial.

Regardless, both functions are subject to Runge's phenomenon (we say they are
Runge functions). The phenomenon is that, due to being interpolated over a set
of equispaced points, as the degree of the interpolating polynomial increases,
the oscillation at the ends of the interval rapidly grows. A direct corollary is
that increasing the degree does not always result in improvement in accuracy.

\section{Appendix}

\begin{figure}
\includegraphics[width=\linewidth]{ex5_f1_5}
\caption{Function $f_1$ from problem 5 with degree 5 polynomial.}
\label{fig:ex5_f1_5}
\end{figure}
\begin{figure}
\includegraphics[width=\linewidth]{ex5_f1_10}
\caption{Function $f_1$ from problem 5 with degree 10 polynomial.}
\label{fig:ex5_f1_10}
\end{figure}
\begin{figure}
\includegraphics[width=\linewidth]{ex5_f1_15}
\caption{Function $f_1$ from problem 5 with degree 15 polynomial.}
\label{fig:ex5_f1_15}
\end{figure}
\begin{figure}
\includegraphics[width=\linewidth]{ex5_f2_5}
\caption{Function $f_2$ from problem 5 with degree 5 polynomial.}
\label{fig:ex5_f2_5}
\end{figure}
\begin{figure}
\includegraphics[width=\linewidth]{ex5_f2_10}
\caption{Function $f_2$ from problem 5 with degree 10 polynomial.}
\label{fig:ex5_f2_10}
\end{figure}
\begin{figure}
\includegraphics[width=\linewidth]{ex5_f1_15}
\caption{Function $f_2$ from problem 5 with degree 15 polynomial.}
\label{fig:ex5_f2_15}
\end{figure}

\begin{figure}
\includegraphics[width=\linewidth]{ex6_f1_5}
\caption{Function $f_1$ from problem 6 with degree 5 polynomial.}
\label{fig:ex6_f1_5}
\end{figure}
\begin{figure}
\includegraphics[width=\linewidth]{ex6_f1_10}
\caption{Function $f_1$ from problem 6 with degree 10 polynomial.}
\label{fig:ex6_f1_10}
\end{figure}
\begin{figure}
\includegraphics[width=\linewidth]{ex6_f1_15}
\caption{Function $f_1$ from problem 6 with degree 15 polynomial.}
\label{fig:ex6_f1_15}
\end{figure}
\begin{figure}
\includegraphics[width=\linewidth]{ex6_f2_5}
\caption{Function $f_2$ from problem 6 with degree 5 polynomial.}
\label{fig:ex6_f2_5}
\end{figure}
\begin{figure}
\includegraphics[width=\linewidth]{ex6_f2_10}
\caption{Function $f_2$ from problem 6 with degree 10 polynomial.}
\label{fig:ex6_f2_10}
\end{figure}
\begin{figure}
\includegraphics[width=\linewidth]{ex6_f1_15}
\caption{Function $f_2$ from problem 6 with degree 15 polynomial.}
\label{fig:ex6_f2_15}
\end{figure}

\end{document}
